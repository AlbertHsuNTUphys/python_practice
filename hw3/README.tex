\documentclass[12pt]{article}

%Packages BEGIN
\usepackage{xcolor}
\usepackage{soul}
\usepackage{listings}
\usepackage{hyperref}
\usepackage{indentfirst}

\lstset{
basicstyle=\small\ttfamily,
breaklines=true
} 

\definecolor{Light}{gray}{.90}
\sethlcolor{Light}

\let\OldTexttt\texttt
\newcommand{\code}[1]{\texttt{\hl{#1}}} 

%Packages END

\title{README for \texttt{Contract Bridge Simulator}, Computer Programming HW3}
\author{B07202020\\Hao-Chien Wang}

\begin{document}
\maketitle

\section{Program Description}%
\label{sec:program_description}

This program simulate a contract bridge game. It doesn't have an AI to play with
the user so it is not really a game. The rules of contract bridge can be found
below, with some definitions of the terms that will be used below (such as 
bids and trumpsuit):\\

\url{https://en.wikipedia.org/wiki/Contract_bridge}  \\

The computer will deal the cards and show the gaming process. The user enters
the bid he wants to bid and the card he want to play. The computer will show
error messages when an invalid operation is done.


\section{Usage}%
\label{sec:usage}
Run \code{python3 main.py}. The players are represented by N, E, S and W to
indicate North, East, South and West, respectively. The game autimatically starts
and deal the cards. N always starts the bidding. During the bidding stage, enter
number between 1 and 7 to indicate levels (when prompted), and use \code{s},
\code{h}, \code{d}, \code{c}, \code{nt} to represent spades, hearts, diamonds
and no trump. Double and redouble are not yet supported. Passing out will
result in an custom exception.\\
\indent When the bidding ended and the final contract is determined, the game
starts. Each player enter a string started by the first letter of a suit and 
followed by the rank to represent the card he want to play (e.g. \code{h4} means
4 of hearts, \code{sa} means ace of spades, \code{d10} means 10 of diamonds,
\code{ck} means king of clubs, etc) . Cards that are
against the rules (e.g. cards that one doesn't own or doesn't follow the leading
suit) are not accepted. The game ends when the declarer succeeds or fail to make
the contract.


\section{Functions and Classes}%
\label{sec:classes}

All functions and classes are defined in \code{ob.py}. \code{main.py} imports
them and run the game. All the functions, classes and their definitions are
listed below.

\subsection{Functions}%
\label{sub:functions}

\noindent \textbf{cardKey}: Defines the value of cards, in order to compare.
When both \code{trumpSute} and \code{leading\_suit} are \code{None}, the function
is used to sort the cards. \code{leadingSuit} can be either \code{Spades},
\code{Hearts}, \code{Diamonds} or \code{Clubs}. \code{trumpSuit} can be all the 
suits above or \code{NoTrump}. The value of a normal card is an integer between
0 and 12, while A has 12 and 2 has 0. A trump suit card has an extra value of
13. Any card that doesn't follow the leading suit has a value of 0.

\subsection{Classes and NamedTuples}%
\label{sub:classes}

\subsubsection{Card}%
\label{ssub:card}
A named tuple representing cards, with slots \code{['rank', 'suit']}.\\
Example: \code{(rank='k', suit='Spades')}

\subsubsection{Bid}%
\label{ssub:bid}
A named tuple representing bids, with slots \code{['level', 'suit']}\\
Example: \code{(level=3, suit='NoTrump')}, \code{(level=4, suit='Diamonds')}


\subsubsection{Hand}%
\label{ssub:hand}

This class represents a hand of cards.\\

\textbf{Attributes}:\\
\begin{itemize}
\item \code{cards}: A dictionary that maps strings of the names of suits to lists
containing the cards.
\end{itemize}

\textbf{Methods}:\\
\begin{itemize}
\item \code{\_\_init\_\_}: The constructor requires a list of card
that is dealt. It is then sorted and saved in a dictionary of lists (self.cards).
\item \code{\_\_str\_\_}: print all the cards in a more readable way.
\item \code{has}: to tell whether this player has cards in a certain suit.
\end{itemize}

\subsubsection{Deck}%
\label{ssub:deck}
This class represents a deck of cards. It can shuffle and deal cards.\\

\textbf{Attributes}:\\
\begin{itemize}
\item \code{\_cards}: A list to save all cards.
\end{itemize}

\textbf{Methods}:\\
\begin{itemize}
\item \code{\_\_init\_\_}: Constructor to generate all cards.
\item \code{shuffle}: Shuffle the cards.
\item \code{deal}: deal cards. return a list of four players (Hand objects)
	with dealt cards. Sorted by N, E, S, W.
\end{itemize}

\subsubsection{Auction}%
\label{ssub:auction}
This class represent the whole bidding process.\\

\textbf{Attributes}
\begin{itemize}
	\item \code{\_bidlist}: A list to save all the bid objects or \code{'pass'}.
	\item \code{\_declarer}: The declarer. Determined after the bidding is done.
		initialized with \code{None}.
	\item \code{\_playerList}: A list to save the hand of the players. Sorted
		by N, E, S, W.
\end{itemize}
\textbf{Methods}
\begin{itemize}
	\item \code{\_contractAccepted}: To tell whether the auction has ended and
		the contract has been determined. That is, a bid is given and all the
		other three players has bid \code{pass}. Returns \code{True} if so, 
		\code{False} otherwise.
	\item \code{\_isLarger}: A function to determine whether a bid is larger 
		than the last non-pass bid. Returns \code{True} if so, \code{False}
		otherwise.
	\item \code{bid}: The method to start bidding. Takes no arguments and 
		return nothing.
	\item \code{declarer}: The method to return the declarer. Returns \code{None}
		if not yet determined.
	\item \code{\_lastBid}: To find the last non-pass bid. Takes no arguments
		and return the bid.
	\item \code{\_newBid}: Starting a new bid. prompts the player to enter
		the needed information. Returns \code{True} if the bid is successful and
		\code{False} if the bid is invalid.

\end{itemize}




\subsubsection{Tricks}%
\label{ssub:tricks}

This class represent the whole playing process, that is, 13 tricks.\\

\textbf{Attributes}
\begin{itemize}
	\item \code{\_scoreTable}: A dictionary that maps \code{'EW'} and \code{'NS'}
		to two lists, storing the tricks they have won.
	\item \code{\_declarer}: A number saving the declarer. 0, 1, 2, 3 means N,
		E, S, W, respectively.
	\item \code{\_playerList}: A list of Hand objects, saving the hands of the
		players.
	\item \code{\_cardTable}: A list saving the cards played in each tricks.
	\item \code{\_trumpSuit}: A string storing the trump suit of the game.
	\item \code{\_contract}: A bid object saving the final contract.
	\item \code{\_\_init\_\_}: 
\end{itemize}

\textbf{Methods}
\begin{itemize}
	\item \code{\_\_init\_\_}: Constructor. Takes an auction abject as an argumen
		and save the required information in the attribute variables.
	\item \code{declarer}: Returns the declarer of the game in terms of a string.
		Takes no arguments.
	\item \code{\_newTrick}: A method to start a new trick. Takes the opener
		and the number of that trick as arguments. It asks each player to play
		a card and returns the player taking the trick.
	\item \code{\_illegalCard}: A method to tell whether a card input is valid.
		Takes the card, the hand of the player playing that card, and the leading
		suit of that trick. If the card follows the leading suit and is in the
		hand, \code{True} is returned. \code{False} otherwise. 
	\item \code{\_playCard}: Asks the player to play a card. Takes the hand of
		the player, the number of the player (0 as N and 3 as W), and the leading
		suit (if the player is not the leader of the suit) as arguments. It calls
		\code{\_illegalCard} to verify the card, then returns the card.
	\item \code{play}: The function to call every tricks, and tell whether the
		game has ended. Takes no argument and returns nothing.
	
	
\end{itemize}

\end{document}
